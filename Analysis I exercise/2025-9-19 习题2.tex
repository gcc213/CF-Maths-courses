% Options for packages loaded elsewhere
\PassOptionsToPackage{unicode}{hyperref}
\PassOptionsToPackage{hyphens}{url}
\documentclass[
  a4paper,
]{ctexart}

\usepackage{xcolor}
\usepackage{amsmath,amssymb}

\usepackage{ctex}
\usepackage{xeCJK}
\setCJKmainfont{SimSun}[BoldFont=SimHei,ItalicFont=KaiTi]


\setcounter{secnumdepth}{-\maxdimen} % remove section numbering

\usepackage{bookmark}
\IfFileExists{xurl.sty}{\usepackage{xurl}}{} % add URL line breaks if available
\urlstyle{same}

\usepackage{iftex}
\ifPDFTeX
  \usepackage[T1]{fontenc}
  \usepackage[utf8]{inputenc}
  \usepackage{textcomp} % provide euro and other symbols
\else % if luatex or xetex
  \defaultfontfeatures{Scale=MatchLowercase}
  \defaultfontfeatures[\rmfamily]{Ligatures=TeX,Scale=1}
\fi
\usepackage{lmodern}

\makeatletter
\@ifundefined{KOMAClassName}{% if non-KOMA class
  \IfFileExists{parskip.sty}{%
    \usepackage{parskip}
  }{% else
    \setlength{\parindent}{0pt}
    \setlength{\parskip}{6pt plus 2pt minus 1pt}}
}{% if KOMA class
  \KOMAoptions{parskip=half}}
\makeatother

\setlength{\emergencystretch}{3em} % prevent overfull lines
\providecommand{\tightlist}{\setlength{\itemsep}{0pt}\setlength{\parskip}{0pt}}

\usepackage{bookmark}
\IfFileExists{xurl.sty}{\usepackage{xurl}}{} % add URL line breaks if available
\urlstyle{same}
\hypersetup{hidelinks,pdfcreator={cjx}}


\usepackage{titling}
% 核心修改:若未手动设置 title,则自动用文件名()作为标题
\title{习题2}
\author{}
\date{2025年9月19日}

% 先加载 enumitem 宏包
\usepackage{enumitem}

% 重新用 enumitem 定义列表样式(此时无命令覆盖)
\usepackage{etoolbox}
\AtBeginEnvironment{enumerate}{
  \setlist[enumerate,2]{
    before={\def\labelenumii{(\arabic{enumii})}},  % 环境开始时执行一次
    itemjoin={\def\labelenumii{(\arabic{enumii})}},  % 每个 \item 前再执行一次(确保覆盖)
    label=(\arabic{enumii})  % 直接指定标签格式(双重保险)
  }
  \apptocmd{\tightlist}{\def\labelenumii{(\arabic{enumii})}}{}{}
}


\usepackage{fancyhdr}

\begin{document}
%%\maketitle
%
\pagestyle{fancy}
\fancyhf{}

\lhead{2025年9月19日}
\chead{习题2}

\renewcommand{\headrulewidth}{1pt}
\rfoot{\thepage}

课上我们已经看到了如何通过Cauchy列的方式定义实数.我们下面从另一种角度来看这件事.这里我们假设我们已经有了整数,有理数,加减乘除,有理数的序关系.

\subsection{第一步}\label{ux7b2cux4e00ux6b65}

\begin{itemize}
\tightlist
\item
  \textbf{定义}
  我们称一个集合\(U\subseteq\mathbb{Q}\)为一个\textbf{分割},如果他满足以下条件:

  \begin{enumerate}
  \def\labelenumi{\arabic{enumi}.}
  \tightlist
  \item
    \(U\)不为空集且\(U \neq \mathbb{Q}\).
  \item
    若\(p \in U\)且\(q<p\),则有\(q \in U\).
  \item
    若\(p \in U\),则存在\(r \in U\)使得\(p<r\).
  \end{enumerate}
\end{itemize}

进一步我们定义全体分割构成的集合为\(\mathbb{R}\).

\begin{enumerate}
\def\labelenumi{\arabic{enumi}.}
\tightlist
\item
  证明:若\(p \in U,q \notin U\),则\(p<q\);若\(p \notin U,p<q\),则\(q \notin U\).
\end{enumerate}

\subsection{第二步}\label{ux7b2cux4e8cux6b65}

\begin{itemize}
\tightlist
\item
  \textbf{定义} 我们称两个分割\(U<V\),若\(U\)是\(V\)的子集.
\end{itemize}

\begin{enumerate}
\def\labelenumi{\arabic{enumi}.}
\setcounter{enumi}{1}
\tightlist
\item
  证明:\(\mathbb{R}\)是一个有序集.
\item
  证明:基于这样一个序关系,\(\mathbb{R}\)有最小上界性质.
\end{enumerate}

\subsection{第三步}\label{ux7b2cux4e09ux6b65}

\begin{itemize}
\tightlist
\item
  \textbf{定义}
  对任意\(U,V \in \mathbb{R}\),我们令\(U+V=\{ p+q \mid p \in U,q \in V \},0=\{ x \in \mathbb{Q} \mid x<0 \}\).
\end{itemize}

\begin{enumerate}
\def\labelenumi{\arabic{enumi}.}
\setcounter{enumi}{3}
\tightlist
\item
  验证上述定义满足加法公理.
\item
  证明:对任意\(U,V,W \in \mathbb{R}\),若\(U<V\),则\(U+W<V+W\).
\end{enumerate}

\subsection{第四步}\label{ux7b2cux56dbux6b65}

\begin{itemize}
\tightlist
\item
  \textbf{定义} \(\mathbb{R}_+=\{ U \in \mathbb{R} \mid U > 0 \}\).
\item
  \textbf{定义}
  对\(U,V \in \mathbb{R}_+,UV=\{ p \mid \exists q \in U,r \in V,q>0,r>0,p \leq qr \}\).特别地,我们定义\(1=\{ q\in\mathbb{Q} \mid q<1 \}\).
\item
  \textbf{定义}
  \(U0=0U=0\);若\(U<0,V<0,UV=(-U)(-V)\);若\(U<0,V>0,UV=-((-U)V)\);若\(U>0,V<0,UV=-(U(-V))\).
\end{itemize}

\begin{enumerate}
\def\labelenumi{\arabic{enumi}.}
\setcounter{enumi}{5}
\tightlist
\item
  验证上述定义满足乘法和除法公理.
\item
  证明:若\(a\neq0\)为有理数,\(b\)为无理数,则\(a+b,ab\)都是无理数.
\item
  证明:\(\sqrt6\)为无理数.
\item
  证明:\(\mathbb{R}\)中有下界集合\(A\)满足\(\inf A=-\sup(-A)\).
\item
  令\(b>1,m,n,p,q \in \mathbb{Z},n,q>0,\frac{m}{n}=\frac{p}{q}\).证明:\((b^m)^\frac{1}{n}=(b^p)^\frac{1}{q}\).
\end{enumerate}

\begin{itemize}
\tightlist
\item
  \textbf{定义}
  若\(x \in \mathbb{R}\),记\(B(x)=\{ b^t \mid t \in \mathbb{Q},t \leq x \}\).定义\(b^x=\sup B(x)\).
\end{itemize}

\begin{enumerate}
\def\labelenumi{\arabic{enumi}.}
\setcounter{enumi}{10}
\tightlist
\item
  证明:\(b^{x+y}=b^xb^y,x,y \in \mathbb{R}\).
\item
  对于\(b>1,y>0\),证明:存在唯一实数\(x\)使得\(b^x=y\).证明可以拆分为以下步骤:

  \begin{enumerate}
  \def\labelenumii{\arabic{enumii}.}
  \tightlist
  \item
    对任意正整数\(n\),有\(b^n-1 \geq n(b-1)\).
  \item
    若\(t>1\)且\(n>\frac{b-1}{t-1}\),则\(b^\frac{1}{n}<t\).
  \item
    若\(w\)使得\(b^w<y\),则当\(n\)足够大时,有\(b^{w+\frac{1}{n}}<y\).
  \item
    若\(w\)使得\(b^w>y\),则当\(n\)足够大时,有\(b^{w-\frac{1}{n}}<y\).
  \item
    令\(A=\{ w \mid b^w<y \}\),证明:\(x=\sup A\)满足\(b^x=y\).
  \item
    证明:上述\(x\)唯一.
  \end{enumerate}
\end{enumerate}

\end{document}
