% Options for packages loaded elsewhere
\PassOptionsToPackage{unicode}{hyperref}
\PassOptionsToPackage{hyphens}{url}
\documentclass[
  a4paper,
]{ctexart}

\usepackage{xcolor}
\usepackage{amsmath,amssymb}

\usepackage{ctex}
\usepackage{xeCJK}
\setCJKmainfont{SimSun}[BoldFont=SimHei,ItalicFont=KaiTi]


\setcounter{secnumdepth}{-\maxdimen} % remove section numbering

\usepackage{bookmark}
\IfFileExists{xurl.sty}{\usepackage{xurl}}{} % add URL line breaks if available
\urlstyle{same}

\usepackage{iftex}
\ifPDFTeX
  \usepackage[T1]{fontenc}
  \usepackage[utf8]{inputenc}
  \usepackage{textcomp} % provide euro and other symbols
\else % if luatex or xetex
  \defaultfontfeatures{Scale=MatchLowercase}
  \defaultfontfeatures[\rmfamily]{Ligatures=TeX,Scale=1}
\fi
\usepackage{lmodern}

\makeatletter
\@ifundefined{KOMAClassName}{% if non-KOMA class
  \IfFileExists{parskip.sty}{%
    \usepackage{parskip}
  }{% else
    \setlength{\parindent}{0pt}
    \setlength{\parskip}{6pt plus 2pt minus 1pt}}
}{% if KOMA class
  \KOMAoptions{parskip=half}}
\makeatother

\setlength{\emergencystretch}{3em} % prevent overfull lines
\providecommand{\tightlist}{\setlength{\itemsep}{0pt}\setlength{\parskip}{0pt}}

\usepackage{bookmark}
\IfFileExists{xurl.sty}{\usepackage{xurl}}{} % add URL line breaks if available
\urlstyle{same}
\hypersetup{hidelinks,pdfcreator={cjx}}


\usepackage{titling}
% 核心修改:若未手动设置 title,则自动用文件名()作为标题
\title{习题3}
\author{}
\date{2025年9月22日}

% 先加载 enumitem 宏包
\usepackage{enumitem}

% 重新用 enumitem 定义列表样式(此时无命令覆盖)
\usepackage{etoolbox}
\AtBeginEnvironment{enumerate}{
  \setlist[enumerate,2]{
    before={\def\labelenumii{(\arabic{enumii})}},  % 环境开始时执行一次
    itemjoin={\def\labelenumii{(\arabic{enumii})}},  % 每个 \item 前再执行一次(确保覆盖)
    label=(\arabic{enumii})  % 直接指定标签格式(双重保险)
  }
  \apptocmd{\tightlist}{\def\labelenumii{(\arabic{enumii})}}{}{}
}


\usepackage{fancyhdr}

\begin{document}
%%\maketitle
%
\pagestyle{fancy}
\fancyhf{}

\lhead{2025年9月22日}
\chead{习题3}

\renewcommand{\headrulewidth}{1pt}
\rfoot{\thepage}

目前我们实数有两种构造(Cauchy列和Dedekind分割),无论你用哪一种,最关键的是得到实数集的确界原理:

\textbf{定理1} 非空有上界集合必有上确界.

这个定理有多种等价表述形式:

\textbf{定理2} 单调有界实数列必收敛.

补充一下收敛的定义:称数列\((a_n)\)收敛到\(L\),若对任意\(\varepsilon > 0\),存在\(N \in \mathbb{N}\)(足够大),使得当\(n \geq N\)时,\(\vert a_n - L \rvert \leq \varepsilon\).我们记为\(\lim\limits_{n \to \infty}a_n=L\).

\begin{enumerate}
\def\labelenumi{\arabic{enumi}.}
\tightlist
\item
  证明:收敛数列必有界.
\item
  证明:设\(\lim\limits_{n \to \infty}s_n=s,\lim\limits_{n \to \infty}t_n=t\),则\(\lim_{n \to \infty}s_nt_n=st\).
\end{enumerate}

\textbf{定理3} 有界实数列必有收敛子列.

\textbf{定理4} Cauchy列必收敛.

\textbf{定理5(闭区间套)}
若\([a_n,b_n] \subseteq \mathbb{R},[a_{n+1},b_{n+1}] \subseteq \mathbb{R}\),则\(\bigcap\limits_{n=1}^\infty[a_n,b_n]\)非空.进一步,若\(\lim\limits_{n \to \infty}\left( a_n-b_n \right)=0\),则上面的集合是单点集.

\textbf{定理6(有限覆盖定理)}
设有一组开区间\(\left( a_\alpha,b_\alpha \right),\alpha \in I\),使得\([a,b]\subseteq\bigcup\limits_{\alpha \in I}\left( a_\alpha,b_\alpha \right)\).则存在\(\alpha_1,\alpha_2,\cdots,\alpha_n \in I,n \in \mathbb{N}\),使得\([a,b] \subseteq \bigcup\limits_{j=1}^n\left( a_\alpha,b_\alpha \right)\).

我们接下来证明它们相互等价.根据定理表述的侧重,可以归类于证明下面三个循环:

\begin{enumerate}
\def\labelenumi{\arabic{enumi}.}
\setcounter{enumi}{2}
\tightlist
\item
  定理1-定理6-定理3-定理2-定理1.
\item
  定理1-定理5-定理2-定理1.
\item
  定理3-定理4-定理3.
\end{enumerate}

\end{document}
